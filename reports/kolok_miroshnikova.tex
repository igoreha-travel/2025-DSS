\documentclass[12pt,a4paper]{article}
\usepackage[utf8]{inputenc}
\usepackage[russian]{babel}
\usepackage[left=2cm,right=2cm,top=2cm,bottom=2cm]{geometry}
\usepackage{graphicx}
\usepackage{amsmath}
\usepackage{amssymb}
\usepackage{booktabs}
\usepackage{array}
\usepackage{caption}
\usepackage{subcaption}
\usepackage{float}
\usepackage{multirow}
\usepackage{minted}
\usepackage{hyperref}

\title{Экспериментальное исследование эффективности нечеткой логики \\ 
для диагностики заболеваний желудка}
\author{}
\date{\today}

\begin{document}

\maketitle

\section*{Гипотеза исследования}
\textbf{Гипотеза:} Нечеткая логическая система, основанная на экспертных правилах о визуальных характеристиках тканей желудка, может эффективно классифицировать образцы на категории "Норма", "Дисплазия", "Рак" с точностью, сопоставимой с упрощенным пороговым методом, но при этом давать более информативную степень уверенности в диагнозе для пограничных случаев.

\section{Методология исследования}

\subsection{Метод 1: Пороговый подход}

Пороговая модель построена по принципу упрощенных клинических протоколов. Каждый из трех параметров оценивается по шкале от 0 до 10, затем вычисляется взвешенная сумма:

\[
S = 0.4 \cdot A \cdot 10 + 0.35 \cdot P \cdot 10 + 0.25 \cdot M \cdot 10 + \varepsilon
\]

где:
\begin{itemize}
    \item $A$ -- атипия ядер (0-10)
    \item $P$ -- полиморфизм клеток (0-10) 
    \item $M$ -- активность митозов (0-10)
    \item $\varepsilon \sim \mathcal{N}(0, 3)$ -- шум, имитирующий вариативность экспертных оценок
\end{itemize}

Для классификации используются жесткие пороги:
\begin{itemize}
    \item $S < 35$: класс "normal"
    \item $35 \leq S < 65$: класс "dysplasia"
    \item $S \geq 65$: класс "cancer"
\end{itemize}

\subsection{Метод 2: Нечеткая логика}

\subsubsection{Архитектура системы}
Система реализует вывод Мамдани с тремя входными и одной выходной переменной:

\begin{itemize}
    \item \textbf{Входные переменные} (шкала 0-10):
    \begin{itemize}
        \item Атипия ядер (\texttt{atypia})
        \item Полиморфизм клеток (\texttt{polymorphism})
        \item Активность митозов (\texttt{mitosis})
    \end{itemize}
    
    \item \textbf{Выходная переменная} (шкала 0-100):
    \begin{itemize}
        \item Диагностический балл (\texttt{diagnosis\_score})
    \end{itemize}
    
    \item \textbf{Термы для каждой переменной}:
    \begin{itemize}
        \item \texttt{low} (низкий)
        \item \texttt{medium} (средний)
        \item \texttt{high} (высокий)
    \end{itemize}
\end{itemize}

\subsubsection{Функции принадлежности}
Использованы гауссовы функции для обеспечения плавных переходов:
\[
\mu_{\text{low}}(x) = e^{-\frac{(x - 1.5)^2}{2 \cdot 1.2^2}}, \quad
\mu_{\text{medium}}(x) = e^{-\frac{(x - 5.0)^2}{2 \cdot 1.5^2}}, \quad
\mu_{\text{high}}(x) = e^{-\frac{(x - 8.5)^2}{2 \cdot 1.2^2}}
\]

\begin{figure}[H]
    \centering
    \includegraphics[width=0.8\textwidth]{функция_принадлежности.png}
    \caption{Функции принадлежности для входных и выходных переменных}
    \label{fig:membership}
\end{figure}

На Рисунке \ref{fig:membership} показаны плавные переходы между термами. В отличие от порогового метода, где границы четкие, здесь значение может одновременно принадлежать нескольким категориям с разной степенью уверенности.

\subsubsection{Экспертные правила}
Система содержит 7 правил, формализующих медицинские знания:

\begin{enumerate}
    \item \textbf{Правило для противоречивых показателей}: Если атипия высокая И митозы низкие ТО средний риск (вес 0.8)
    \item \textbf{Правило для неопределенных случаев}: Если все показатели средние ТО средний риск (вес 1.0)
    \item \textbf{Правило для слабо выраженной патологии}: Если атипия низкая И полиморфизм средний И митозы низкие ТО низкий риск (вес 0.6)
    \item \textbf{Правило для явной патологии}: Если все показатели высокие ТО высокий риск (вес 0.9)
    \item \textbf{Правило для нормы}: Если все показатели низкие ТО низкий риск (вес 0.9)
    \item \textbf{Правило для выраженных изменений}: Если атипия высокая И полиморфизм высокий И митозы средние ТО высокий риск (вес 0.7)
    \item \textbf{Правило для изолированного признака}: Если атипия высокая И остальные низкие ТО средний риск (вес 0.8)
\end{enumerate}

\subsubsection{Процесс вывода}
\begin{enumerate}
    \item \textbf{Фаззификация}: Перевод входных значений в степени принадлежности
    \item \textbf{Активация правил}: Вычисление степени выполнения каждого правила
    \item \textbf{Агрегация}: Объединение результатов всех правил
    \item \textbf{Дефаззификация}: Метод центра тяжести для получения итогового балла
\end{enumerate}

\subsection{Генерация тестовых данных}

Для оценки моделей создан синтетический датасет образцов с тремя типами случаев:

\subsubsection{Четкие случаи (36\%)}
\begin{itemize}
    \item \textbf{Норма}: все параметры < 3
    \item \textbf{Рак}: все параметры > 7
\end{itemize}

\subsubsection{Пограничные случаи (36\%)}
\begin{itemize}
    \item Все параметры в диапазоне 3-7
    \item Случайное распределение между классами
    \item Имитируют диагностически сложные случаи
\end{itemize}

\subsubsection{Противоречивые случаи (28\%)}
\begin{itemize}
    \item \textbf{Высокая атипия + низкие митозы}: атипия > 7, остальные < 4
    \item \textbf{Смешанная картина}: разнонаправленные изменения параметров
    \item Требуют экспертной интерпретации
\end{itemize}

\begin{figure}[H]
    \centering
    \includegraphics[width=0.8\textwidth]{3д.png}
    \caption{3D пространство признаков с распределением образцов}
    \label{fig:3d_features}
\end{figure}

На Рисунке \ref{fig:3d_features} видно распределение образцов в пространстве признаков. Зеленые точки (норма) сосредоточены в области низких значений, красные (рак) -- в области высоких значений, оранжевые (дисплазия) занимают промежуточную зону с частичным перекрытием.

\section{Результаты и анализ}

\subsection{Матрица ошибок нечеткой системы}

\begin{figure}[H]
    \centering
    \includegraphics[width=0.8\textwidth]{матрица_ошибок.png}
    \caption{Матрица ошибок для нечеткой системы}
    \label{fig:confusion_matrix}
\end{figure}

На Рисунке \ref{fig:confusion_matrix} представлена матрица ошибок нечеткой системы:
\begin{itemize}
    \item \textbf{Диагональ}: правильно классифицированные случаи
    \item \textbf{Вне диагонали}: ошибки классификации
    \item Ошибки: 23 случая нормальных данных ошибочно отнесены к дисплазии
    \item 30 случаев дисплазии ошибочно отнесены к норме
    \item 10 случаев дисплазии ошибочно отнесены к раку
\end{itemize}

Матрица показывает, что система чаще всего ошибается, путая дисплазию с соседними классами, что отражает объективную сложность классификации пограничных состояний.

\subsection{Распределение диагностических баллов}

\begin{figure}[H]
    \centering
    \includegraphics[width=0.7\textwidth]{распределение_баллов1.png}
    \caption{Распределение предсказанных баллов по классам}
    \label{fig:score_distribution1}
\end{figure}

\begin{figure}[H]
    \centering
    \includegraphics[width=0.7\textwidth]{распределение_баллов2.png}
    \caption{Распределение предсказанных баллов по классам}
    \label{fig:score_distribution2}
\end{figure}

На Рисунках \ref{fig:score_distribution1} и \ref{fig:score_distribution2} показаны гистограммы распределения предсказанных баллов для каждого истинного класса:

\subsubsection{Класс «cancer»}
\begin{itemize}
    \item Основная масса: 80-100 баллов
    \item Распределение смещено вправо, что соответствует высокой уверенности в диагнозе
\end{itemize}

\subsubsection{Класс «dysplasia»}
\begin{itemize}
    \item Бимодальное распределение: пики в зонах 0-10 и 50 баллов
    \item Широкий разброс: от 0 до 80 баллов
    \item Отражает неоднородность категории "дисплазия"
\end{itemize}

\subsubsection{Класс «normal»}
\begin{itemize}
    \item Узкое распределение: 10-50 баллов
    \item Пик в зоне 10-20 баллов
    \item Мало случаев в диапозоне 20-40 баллов
\end{itemize}

Бимодальность распределения для дисплазии особенно показательна: она демонстрирует, что эта категория действительно содержит два подтипа -- ближе к норме и ближе к раку.

\subsection{Сравнительный анализ моделей}

\begin{figure}[H]
    \centering
    \includegraphics[width=0.9\textwidth]{сравнение_моделей.png}
    \caption{Сравнение эффективности моделей на разных типах случаев}
    \label{fig:model_comparison}
\end{figure}

На Рисунке \ref{fig:model_comparison} представлено сравнение двух подходов:

\subsubsection{Четкие случаи}
\begin{itemize}
    \item Пороговая модель: 88.9\% точности
    \item Нечеткая система: 69.7\% точности
    \item Разница: +19.2\% в пользу пороговой модели
\end{itemize}

\subsubsection{Пограничные случаи}
\begin{itemize}
    \item Пороговая модель: 82.5\% точности
    \item Нечеткая система: 63.5\% точности
    \item Разница: +19.0\% в пользу пороговой модели
\end{itemize}

\subsection{Анализ пограничных случаев}

Система выявила 16 пограничных случая (8\% от общего числа), где:
\begin{itemize}
    \item Диагностический балл находился в диапазоне 0-20 (граница норма/дисплазия)
    \item Или в диапазоне 50-80 (граница дисплазия/рак)
\end{itemize}

\section{Обсуждение результатов}

\subsection{Сравнительная таблица моделей}

\begin{table}[H]
\centering
\caption{Сравнительные характеристики моделей}
\label{tab:model_comparison}
\begin{tabular}{p{5cm}cc}
\toprule
\textbf{Характеристика} & \textbf{Пороговая модель} & \textbf{Нечеткая система} \\
\midrule
Общая точность & \textbf{88.9\%} & 69.7\% \\
Точность на пограничных случаях & \textbf{82.5\%} & 63.5\% \\
Интерпретируемость решений & Низкая & \textbf{Высокая} \\
Возможность объяснения & Нет & \textbf{Да, через правила} \\
Учет противоречивых признаков & Нет & \textbf{Да} \\
Плавность переходов & Нет, резкие границы & \textbf{Да, плавные} \\
Требуемые экспертные знания & Минимальные & \textbf{Значительные} \\
\bottomrule
\end{tabular}
\end{table}

\subsection{Анализ причин расхождений}

\subsubsection{Преимущества пороговой модели}
\begin{enumerate}
    \item \textbf{Простота}: Линейная комбинация легко оптимизируется
    \item \textbf{Стабильность}: Меньше параметров для настройки
    \item \textbf{Предсказуемость}: Поведение системы легко понять
    \item \textbf{Эффективность на четких случаях}: Когда признаки однозначны, простые методы работают хорошо
\end{enumerate}

\subsubsection{Недостатки нечеткой системы в текущей реализации}
\begin{enumerate}
    \item \textbf{Неполная база правил}: 7 правил недостаточно для охвата всех возможных комбинаций
    \item \textbf{Отсутствие обучения}: Веса правил и параметры функций заданы априорно
    \item \textbf{Ограниченность экспертных знаний}: Правила основаны на общих принципах, а не на конкретных клинических данных
    \item \textbf{Проблема масштабирования}: Каждое новое правило требует экспертной валидации
\end{enumerate}

\subsection{Качественные преимущества нечеткого подхода}

Несмотря на более низкую количественную точность, нечеткая система демонстрирует важные качественные преимущества:

\subsubsection{Интерпретируемость}
Каждое решение может быть объяснено через активированные правила:
\begin{itemize}
    \item "Система оценила риск в 54 балла, потому что активировались правила №2 и №7"
    \item "Степень уверенности в дисплазии: 0.7, в раке: 0.3"
\end{itemize}

\subsubsection{Плавность оценок}
\begin{itemize}
    \item Значение 64.9 и 65.1 дают схожие оценки риска
    \item Отсутствие резких скачков при малых изменениях параметров
    \item Более соответствует непрерывной природе биологических процессов
\end{itemize}

\subsubsection{Устойчивость к противоречиям}
\begin{itemize}
    \item Противоречивые признаки не "ломают" систему
    \item Частичная активация противоречивых правил
    \item Результат отражает неопределенность ситуации
\end{itemize}

\section{Выводы и заключение}

\subsection{Оценка гипотезы}

\textbf{Гипотеза не подтвердилась в своей количественной части:}
\begin{itemize}
    \item Нечеткая система показала значимо более низкую точность (69.7\% против 88.9\%)
    \item На пограничных случаях разница составила 19\%
    \item Пороговая модель оказалась эффективнее по метрикам точности
\end{itemize}

\textbf{Гипотеза частично подтвердилась в качественной части:}
\begin{itemize}
    \item Нечеткая система действительно обеспечивает более информативные оценки
    \item Дает степени уверенности вместо бинарных решений
    \item Позволяет объяснять решения через правила
    \item Обеспечивает плавные переходы между состояниями
\end{itemize}

\subsection{Ограничения исследования}

\begin{enumerate}
    \item \textbf{Синтетические данные}: Реальные гистологические данные могут иметь другую структуру
    \item \textbf{Упрощенная модель заболеваний}: Рассмотрены только три параметра из многих возможных
    \item \textbf{Ограниченная база правил}: 7 правил недостаточно для полноценной экспертной системы
    \item \textbf{Отсутствие реальных экспертных знаний}: Правила созданы на основе общих принципов, а не клинического опыта
    \item \textbf{Фиксированные параметры}: Функции принадлежности не настраивались на данных
\end{enumerate}

\subsection{Перспективы улучшения}

Для повышения эффективности нечеткой системы требуются:

\subsubsection{Расширение экспертной базы}
\begin{itemize}
    \item Увеличение количества правил до 20-30
    \item Привлечение врачей-патологов для формулирования правил
    \item Учет дополнительных параметров (размер ядер, структура хроматина и др.)
\end{itemize}

\subsubsection{Технические улучшения}
\begin{itemize}
    \item Настройка параметров функций принадлежности на данных
    \item Реализация адаптивного механизма корректировки весов правил
    \item Интеграция с методами машинного обучения для обучения параметров
\end{itemize}

\subsubsection{Валидация на реальных данных}
\begin{itemize}
    \item Тестирование на аннотированных гистологических изображениях
    \item Сравнение с решениями нескольких независимых экспертов
    \item Оценка клинической полезности в реальных условиях
\end{itemize}

\subsection{Заключительные замечания}

Данное исследование демонстрирует, что переход от простых пороговых методов к сложным экспертным системам на основе нечеткой логики требует значительных усилий. Текущая реализация показала, что \textbf{простота зачастую эффективнее сложности}, когда речь идет о количественных метриках точности.

Однако для задач медицинской диагностики, где важны не только итоговые метрики, но и интерпретируемость, объяснимость и способность работать с неопределенностью, нечеткий подход сохраняет свою актуальность. Ключевым условием его успешного применения является \textbf{глубокое вовлечение экспертов-медиков} в процесс разработки и валидации системы.

Текущая работа может рассматриваться как первый шаг в создании полноценной системы поддержки принятия решений для патологоанатомов, требующий дальнейшего развития и совершенствования.

\end{document}