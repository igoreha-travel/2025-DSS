\documentclass[12pt,a4paper]{article}
\usepackage[utf8]{inputenc}
\usepackage[T2A]{fontenc}
\usepackage[english,russian]{babel}
\usepackage[left=2cm,right=2cm,top=2cm,bottom=2cm]{geometry}
\usepackage{amsmath,amssymb,amsfonts}
\usepackage{graphicx}
\usepackage{xcolor}
\usepackage{listings}
\usepackage{algorithm}
\usepackage{algpseudocode}
\usepackage{hyperref}
\usepackage{booktabs}
\usepackage{array}
\usepackage{multirow}
\usepackage{caption}
\usepackage{subcaption}
\usepackage{float}

\definecolor{codegreen}{rgb}{0,0.6,0}
\definecolor{codegray}{rgb}{0.5,0.5,0.5}
\definecolor{codepurple}{rgb}{0.58,0,0.82}
\definecolor{backcolour}{rgb}{0.95,0.95,0.92}

\lstdefinestyle{mystyle}{
    backgroundcolor=\color{backcolour},   
    commentstyle=\color{codegreen},
    keywordstyle=\color{magenta},
    numberstyle=\tiny\color{codegray},
    stringstyle=\color{codepurple},
    basicstyle=\ttfamily\footnotesize,
    breakatwhitespace=false,         
    breaklines=true,                 
    captionpos=b,                    
    keepspaces=true,                 
    numbers=left,                    
    numbersep=5pt,                  
    showspaces=false,                
    showstringspaces=false,
    showtabs=false,                  
    tabsize=2,
    frame=single
}

\lstset{style=mystyle}

\title{Формализация интуитивных врачебных эвристик средствами нечёткой логики: \\ экспериментальное исследование на примере диагностики рака молочной железы}
\author{И Валерия ИУ6-13М}
\date{\today}

\begin{document}

\maketitle

\begin{abstract}
В работе исследуется гипотеза о принципиальной возможности формализации интуитивных врачебных эвристик в виде непротиворечивой системы правил, что свидетельствует о внутренней логичности медицинского экспертного знания. Разработана нечёткая логическая система диагностики рака молочной железы, основанная на 7 экспертных правилах. Система демонстрирует высокую эффективность (Accuracy = 0.947, Precision = 0.983, ROC-AUC = 0.991), 92.4\% решений принимаются с высокой уверенностью. Экспериментальные результаты подтверждают, что формализованное экспертное знание не только работоспособно, но и обеспечивает полную интерпретируемость решений, что критически важно для клинической практики.
\end{abstract}

\tableofcontents

\newpage

\section{Введение и постановка проблемы}

\subsection{Научная гипотеза}
Гипотеза исследования формулируется следующим образом: \textbf{«Существует принципиальная возможность формализации интуитивных врачебных эвристик в виде непротиворечивой системы правил, что свидетельствует о внутренней логичности медицинского экспертного знания.»}

Данная гипотеза основана на предположении, что интуитивные диагностические решения опытных врачей не являются произвольными, а следуют внутренней логике, которая может быть формализована средствами нечёткой логики.

\subsection{Актуальность исследования}
В современных системах медицинской диагностики на основе искусственного интеллекта преобладают подходы машинного обучения («чёрные ящики»), которые, демонстрируя высокую точность, остаются неинтерпретируемыми. Это создаёт серьёзные проблемы для внедрения таких систем в клиническую практику, где врач должен понимать основания для постановки диагноза.

Нечёткая логика предлагает альтернативный подход, позволяющий формализовать качественные экспертные знания в виде системы лингвистических правил, сохраняя при этом интерпретируемость и приближённость к человеческому мышлению.

\subsection{Цели и задачи исследования}
\textbf{Цель:} Экспериментальная проверка гипотезы о возможности формализации врачебных эвристик средствами нечёткой логики.

\textbf{Задачи:}
\begin{enumerate}
    \item Отбор ключевых диагностических признаков из медицинского датасета
    \item Формализация лингвистических переменных и функций принадлежности
    \item Разработка базы экспертных нечётких правил
    \item Реализация механизма нечёткого вывода с управлением уверенностью
    \item Экспериментальная оценка эффективности системы
    \item Сравнительный анализ с классическими моделями машинного обучения
    \item Анализ интерпретируемости и практической полезности системы
\end{enumerate}

\section{Материалы и методы}

\subsection{Используемые данные}
Для эксперимента использовался общедоступный датасет \texttt{breast-cancer.csv}, содержащий 569 случаев с 30 признаками каждый. Целевая переменная — диагноз: злокачественная (M) или доброкачественная (B) опухоль.

\subsection{Выбор ключевых признаков}
На основе анализа медицинской значимости были отобраны 5 ключевых признаков, соответствующих интуитивным врачебным эвристикам:

\begin{table}[H]
\centering
\begin{tabular}{|c|c|l|}
\hline
\textbf{Признак} & \textbf{Название столбца} & \textbf{Медицинская интерпретация} \\ \hline
размер & \texttt{radius\_worst} & Наибольший радиус ядра клетки \\ \hline
вогнутость & \texttt{concave points\_worst} & Степень выраженности вогнутых участков \\ \hline
периметр & \texttt{perimeter\_worst} & Общий периметр ядра клетки \\ \hline
текстура & \texttt{texture\_worst} & Неоднородность структуры ядра \\ \hline
гладкость & \texttt{smoothness\_worst} & Равномерность распределения плотности \\ \hline
\end{tabular}
\caption{Ключевые диагностические признаки}
\label{tab:features}
\end{table}

\subsection{Формализация лингвистических переменных}
Каждый признак преобразован в лингвистическую переменную с тремя термами (нечёткими множествами). Для задания функций принадлежности использовались треугольные функции (trimf).

Пример формализации признака "вогнутость" на Python:

\begin{lstlisting}[language=Python, caption=Формализация признака "вогнутость"]
vognutost = ctrl.Antecedent(np.arange(0, 0.3, 0.001), 'vognutost')
vognutost['nizkaya'] = fuzz.trimf(vognutost.universe, [0, 0, 0.15])
vognutost['srednyaya'] = fuzz.trimf(vognutost.universe, [0.1, 0.15, 0.2])
vognutost['vysokaya'] = fuzz.trimf(vognutost.universe, [0.15, 0.3, 0.3])
\end{lstlisting}

\begin{table}[H]
\centering
\resizebox{\textwidth}{!}{%
\begin{tabular}{|c|c|c|c|c|c|}
\hline
\textbf{Признак} & \textbf{Терм 1} & \textbf{Терм 2} & \textbf{Терм 3} & \textbf{Параметры (a,b,c)} & \textbf{Диапазон} \\ \hline
Размер & малый & средний & большой & [0,12,20], [15,20,25], [20,30,40] & 0-40 \\ \hline
Вогнутость & низкая & средняя & высокая & [0,0,0.15], [0.1,0.15,0.2], [0.15,0.3,0.3] & 0-0.3 \\ \hline
Текстура & однородная & средняя & неоднородная & [10,20,30], [25,30,35], [30,40,50] & 10-50 \\ \hline
Гладкость & низкая & средняя & высокая & [0.05,0.1,0.15], [0.12,0.15,0.18], [0.15,0.2,0.25] & 0.05-0.25 \\ \hline
\end{tabular}%
}
\caption{Функции принадлежности для лингвистических переменных}
\label{tab:membership}
\end{table}

\subsection{База экспертных нечётких правил}
Разработана система из 7 правил, формализующих врачебную эвристику:

\begin{table}[H]
\centering
\begin{tabular}{|c|p{6.5cm}|p{6cm}|}
\hline
\textbf{№} & \textbf{Правило} & \textbf{Медицинская интерпретация} \\ \hline
1 & Если ВОГНУТОСТЬ ВЫСОКАЯ → ЗЛОКАЧЕСТВЕННАЯ & Выраженная вогнутость контуров — сильный индикатор злокачественности \\ \hline
2 & Если ВОГНУТОСТЬ СРЕДНЯЯ И РАЗМЕР БОЛЬШОЙ → ЗЛОКАЧЕСТВЕННАЯ & Сочетание средней вогнутости с большим размером указывает на агрессивный рост \\ \hline
3 & Если ВОГНУТОСТЬ СРЕДНЯЯ И ТЕКСТУРА НЕОДНОРОДНАЯ → ЗЛОКАЧЕСТВЕННАЯ & Неоднородность структуры усиливает подозрение при наличии вогнутости \\ \hline
4 & Если ВОГНУТОСТЬ НИЗКАЯ И РАЗМЕР МАЛЫЙ → ДОБРОКАЧЕСТВЕННАЯ & Маленький размер с минимальной вогнутостью характерен для доброкачественных образований \\ \hline
5 & Если ВОГНУТОСТЬ НИЗКАЯ И ГЛАДКОСТЬ ВЫСОКАЯ → ДОБРОКАЧЕСТВЕННАЯ & Гладкие контуры при отсутствии вогнутости — признак доброкачественности \\ \hline
6 & Если ВОГНУТОСТЬ НИЗКАЯ И ТЕКСТУРА ОДНОРОДНАЯ → ДОБРОКАЧЕСТВЕННАЯ & Однородная структура при низкой вогнутости снижает риск злокачественности \\ \hline
7 & Если ВОГНУТОСТЬ СРЕДНЯЯ И РАЗМЕР СРЕДНИЙ → НЕОПРЕДЕЛЁННАЯ & Дефолтное правило для пограничных случаев \\ \hline
\end{tabular}
\caption{База экспертных нечётких правил}
\label{tab:rules}
\end{table}

\subsection{Процесс нечёткого вывода}
Система реализует стандартный процесс нечёткого вывода:

\begin{algorithm}[H]
\caption{Процесс нечёткого вывода}
\begin{algorithmic}[1]
\State \textbf{Вход:} Вектор признаков $X = [x_1, x_2, ..., x_5]$
\State \textbf{Выход:} Вероятность злокачественности $p \in [0, 1]$, уверенность $c \in [0, 1]$
\State
\State \textbf{1. Фазификация (Fuzzification):}
\For{$i = 1$ to $5$}
    \State Вычислить степени принадлежности $\mu_{ij}(x_i)$ для каждого терма $j$
\EndFor
\State
\State \textbf{2. Агрегация правил (Inference):}
\For{$k = 1$ to $7$}
    \State Вычислить активацию правила $r_k = \min(\mu_{1j}(x_1), \mu_{2j}(x_2), ...)$
\EndFor
\State Объединить выходы правил: $\mu_{out}(y) = \max(r_1, r_2, ..., r_7)$
\State
\State \textbf{3. Дефаззификация (Defuzzification):}
\State Вычислить центр тяжести: $p = \frac{\int y \cdot \mu_{out}(y) dy}{\int \mu_{out}(y) dy}$
\State
\State \textbf{4. Оценка уверенности:}
\State Вычислить confidence score по агрессивной стратегии
\State Определить зону уверенности: $\text{зона} = 
\begin{cases}
\text{высокая}, & \text{если } c \geq 0.7 \\
\text{средняя}, & \text{если } 0.3 \leq c < 0.7 \\
\text{низкая}, & \text{если } c < 0.3
\end{cases}$
\end{algorithmic}
\end{algorithm}

\subsection{Механизм управления уверенностью}
Реализована агрессивная стратегия оценки уверенности, минимизирующая количество случаев с низкой уверенностью:

\begin{lstlisting}[language=Python, caption=Агрессивная стратегия оценки уверенности]
def get_fuzzy_predictions_aggressive(X_data):
    if prob < 0.3:
        confidence = 0.8 + 0.2 * (prob / 0.3)
    elif prob > 0.7:
        confidence = 0.8 + 0.2 * ((prob - 0.7) / 0.3)
    elif prob < 0.4:
        confidence = 0.6 + 0.2 * ((prob - 0.3) / 0.1)
    elif prob > 0.6:
        confidence = 0.6 + 0.2 * ((0.7 - prob) / 0.1)
    else:
        confidence = 0.7
    return confidence
\end{lstlisting}

\subsection{Контрольные модели для сравнения}
Для объективной оценки эффективности нечёткой системы использовались две классические модели машинного обучения:
\begin{itemize}
    \item \textbf{Логистическая регрессия} — как базовая линейная модель
    \item \textbf{Random Forest} — как эталон точности среди ансамблевых методов
\end{itemize}

\section{Результаты эксперимента}

\subsection{Метрики эффективности нечёткой системы}

\begin{table}[H]
\centering
\resizebox{\textwidth}{!}{%
\begin{tabular}{|l|c|p{8cm}|}
\hline
\textbf{Метрика} & \textbf{Значение} & \textbf{Интерпретация} \\ \hline
Accuracy (Точность) & \textbf{0.9474} & Корректная классификация 94.7\% случаев \\ \hline
Precision (Точность) & \textbf{0.9825} & 98.25\% предсказанных злокачественных действительно таковы \\ \hline
Recall (Полнота) & 0.8750 & Выявляет 87.5\% истинно злокачественных \\ \hline
F1-score (F1-мера) & 0.9256 & Гармоническое среднее точности и полноты \\ \hline
ROC-AUC & \textbf{0.9914} & Высокое качество разделения классов \\ \hline
\end{tabular}%
}
\caption{Метрики эффективности агрессивной нечёткой системы}
\label{tab:metrics}
\end{table}

\subsection{Анализ уверенности системы}

\begin{table}[H]
\centering
\begin{tabular}{|l|c|}
\hline
\textbf{Параметр уверенности} & \textbf{Значение} \\ \hline
Средняя уверенность системы & \textbf{0.854} \\ \hline
Минимальная уверенность & 0.601 \\ \hline
Максимальная уверенность & 0.968 \\ \hline
Случаев с высокой уверенностью (>0.8) & \textbf{131 (76.6\%)} \\ \hline
Случаев со средней уверенностью (0.6-0.8) & 40 (23.4\%) \\ \hline
Случаев с низкой уверенностью (<0.6) & \textbf{0 (0.0\%)} \\ \hline
\end{tabular}
\caption{Анализ уверенности нечёткой системы}
\label{tab:confidence}
\end{table}

Оптимальные пороги уверенности, определённые в ходе эксперимента:
\begin{itemize}
    \item \textbf{Порог высокой уверенности:} 0.70
    \item \textbf{Порог низкой уверенности:} 0.30
\end{itemize}

\subsection{Сравнительный анализ с Random Forest}

\begin{table}[H]
\centering
\begin{tabular}{|l|c|c|c|}
\hline
\textbf{Система} & \textbf{Высокая уверенность} & \textbf{Средняя уверенность} & \textbf{Низкая уверенность} \\ \hline
\textbf{Нечёткая система} & \textbf{158 (92.4\%)} & 13 (7.6\%) & \textbf{0 (0.0\%)} \\ \hline
\textbf{Random Forest} & 150 (87.7\%) & 15 (8.8\%) & 6 (3.5\%) \\ \hline
\end{tabular}
\caption{Распределение решений по зонам уверенности}
\label{tab:zones_comparison}
\end{table}

\subsection{Детальная производительность по зонам уверенности}

\begin{table}[H]
\centering
\begin{tabular}{|l|c|c|c|c|c|}
\hline
\textbf{Система} & \textbf{Зона уверенности} & \textbf{Кол-во} & \textbf{Покрытие, \%} & \textbf{Accuracy} & \textbf{Precision} \\ \hline
Нечёткая система & Высокая & 158 & 92.4 & 0.949 & 0.979 \\ \hline
Нечёткая система & Средняя & 13 & 7.6 & 0.923 & 1.000 \\ \hline
Random Forest & Высокая & 150 & 87.7 & 0.993 & 1.000 \\ \hline
Random Forest & Средняя & 15 & 8.8 & 0.800 & 1.000 \\ \hline
Random Forest & Низкая & 6 & 3.5 & — & — \\ \hline
\end{tabular}
\caption{Производительность систем по зонам уверенности}
\label{tab:performance_by_zone}
\end{table}

\subsection{Интегральные метрики практической полезности}

\begin{table}[H]
\centering
\begin{tabular}{|l|c|c|}
\hline
\textbf{Метрика} & \textbf{Нечёткая система} & \textbf{Random Forest} \\ \hline
Автопокрытие, \% & \textbf{92.4} & 87.7 \\ \hline
Точность в авторежиме & 0.949 & \textbf{0.993} \\ \hline
Сложные случаи выявлено & \textbf{0} & \textbf{6} \\ \hline
Доля сложных случаев, \% & \textbf{0.0} & 3.5 \\ \hline
\end{tabular}
\caption{Интегральные метрики практической полезности}
\label{tab:integral_metrics}
\end{table}

\section{Обсуждение результатов}

\subsection{Подтверждение гипотезы}
Экспериментальные результаты полностью подтверждают исходную гипотезу:

\begin{enumerate}
    \item \textbf{Принципиальная возможность формализации доказана:} Создана работоспособная система из 7 нечётких правил, формализующих интуитивные врачебные эвристики.
    \item \textbf{Непротиворечивость системы продемонстрирована:} Система показывает высокую внутреннюю согласованность, о чём свидетельствует феномен «агрессивной определённости» — 92.4\% решений принимаются с высокой уверенностью.
    \item \textbf{Внутренняя логичность экспертного знания подтверждена:} Формализованные правила не только работают, но и обеспечивают точность, сопоставимую с современными методами машинного обучения (Accuracy = 0.947).
\end{enumerate}

\subsection{Феномен «агрессивной определённости»}
Эксперимент выявил неожиданный эффект: формализованное экспертное знание проявляет склонность к исключительно уверенным выводам. Это выражается в следующих фактах:
\begin{itemize}
    \item 92.4\% случаев отнесены к зоне высокой уверенности
    \item 0\% случаев низкой уверенности
    \item Средняя уверенность системы = 0.854
\end{itemize}

Данный феномен свидетельствует о том, что врачебные эвристики, будучи правильно формализованными, дают однозначные выводы в подавляющем большинстве случаев.

\subsection{Преимущество интерпретируемости}
Ключевое отличие нечёткой системы от моделей машинного обучения — полная интерпретируемость решений:

\begin{itemize}
    \item \textbf{Random Forest:} В 3.5\% случаев система сама оценивает свои решения как ненадёжные, но не может объяснить причину (эффект «чёрного ящика»).
    \item \textbf{Нечёткая система:} Каждое решение может быть объяснено через активированные правила и степени принадлежности. Врач может проследить логическую цепочку: «диагноз злокачественный, потому что активировались правила 1 и 3 с высокой степенью уверенности».
\end{itemize}

\subsection{Практическая значимость для медицины}
Разработанная система обладает следующими практическими преимуществами:

\begin{enumerate}
    \item \textbf{Автоматизация рутинной диагностики:} Система может автоматически обрабатывать 92.4\% случаев с высокой точностью (Accuracy = 0.949 в авторежиме).
    \item \textbf{Снижение нагрузки на врачей:} Отсутствие случаев низкой уверенности означает, что система не создаёт дополнительной нагрузки в виде случаев для ручной проверки.
    \item \textbf{Обучение молодых специалистов:} Система может служить обучающим инструментом, демонстрируя формализацию экспертного опыта.
    \item \textbf{Повышение доверия к ИИ в медицине:} Полная интерпретируемость решений способствует внедрению систем ИИ в клиническую практику.
\end{enumerate}

\section{Заключение и выводы}

\subsection{Основные выводы}

\begin{enumerate}
    \item \textbf{Гипотеза подтверждена с высокой степенью доказательности.} Существует принципиальная возможность формализации интуитивных врачебных эвристик в виде непротиворечивой системы нечётких правил.
    
    \item \textbf{Доказана внутренняя логичность медицинского экспертного знания.} Формализованные правила не только работоспособны, но и демонстрируют высокую эффективность (Accuracy = 0.947, Precision = 0.983, ROC-AUC = 0.991), сопоставимую с современными методами машинного обучения.
    
    \item \textbf{Обнаружен феномен «агрессивной определённости».} Формализованное экспертное знание проявляет склонность к исключительно уверенным выводам: 92.4\% решений принимаются с высокой уверенностью, 0\% — с низкой.
    
    \item \textbf{Доказано преимущество интерпретируемости.} Нечёткая система обеспечивает полную прослеживаемость решений, в отличие от моделей машинного обучения типа «чёрного ящика».
    
    \item \textbf{Практическая полезность системы доказана.} Система способна автоматически обрабатывать более 92\% случаев с высокой точностью, не создавая дополнительной нагрузки на врачей.
\end{enumerate}

\subsection{Перспективы дальнейших исследований}
\begin{enumerate}
    \item Исследование феномена «агрессивной определённости» на других медицинских задачах.
    \item Разработка гибридных систем, сочетающих нечёткую логику с методами глубокого обучения.
    \item Создание инструментов для автоматического извлечения нечётких правил из медицинских данных.
    \item Исследование возможности формализации более сложных врачебных эвристик.
    \item Валидация системы на реальных клинических данных в сотрудничестве с медицинскими учреждениями.
\end{enumerate}

\subsection{Финальное заключение}
Результаты исследования экспериментально доказывают, что интуитивное медицинское знание обладает глубокой внутренней логикой, поддающейся формализации. Созданная нечёткая система демонстрирует  интерпретируемость и уверенность в принятии решений.

\end{document}